\documentclass[12pt]{report}
\usepackage[dvipsnames]{xcolor}
\usepackage{amssymb,amsmath,graphicx}
\usepackage{hyperref}
\usepackage{tcolorbox}
\usepackage{tikz, appendix}
\usetikzlibrary{shapes,arrows}
\usetikzlibrary{calc}
\usetikzlibrary{positioning, decorations.pathmorphing, shadows,decorations.markings}
\usepackage{pgf,pgfarrows,pgfnodes,pgfautomata,pgfheaps,pgfshade,tikz}
\usepackage{soul}
\parindent=0pt
\parskip=4pt
\textwidth = 7in
\oddsidemargin = -.3in
\topmargin = -.8in
\textheight = 9.4in
\usepackage{empheq}
\usepackage{fontenc}
\usepackage{tikz}
\usepackage{pgfplots}
\usepackage{mathtools}
\usepackage{xfrac}
\usepackage{geometry}
\usepackage{xcolor}
\usepackage{physics}
\parindent=0pt
\parskip=16pt
\usepgfplotslibrary{fillbetween}
\newcommand*\widefbox[1]{\fbox{\hspace{2em}#1\hspace{2em}}}
\newcommand{\remarks}[1]{{\leavevmode\color{blue} #1}}
\newcommand{\examples}[1]{{\leavevmode\color{red} #1}}
\newcommand{\notes}[1]{{\leavevmode\color{OliveGreen} #1}}
\newcommand{\Mod}[1]{\(\mathrm{mod}\#1)}
\DeclareMathOperator{\rref}{rref}
\DeclareMathOperator{\nullity}{nullity}
\DeclareMathOperator{\Span}{Span}	
\DeclareMathOperator{\Nul}{Nul}
\parindent=0pt
\parskip=4pt
\textwidth = 7in
\oddsidemargin = -.3in
\topmargin = -.8in
\textheight = 9.4in

\begin{document}
	\begin{center} \textbf{\textsc{Math 171 \ \hfill Day 1 - Group definitions \hfill Abstract Algebra}} \\ \vspace{-.1in} \rule{\linewidth}{.01in} \end{center} \vspace{-.15in} \textsc{Prof. Fukshansky}  \hfill \vskip .2in
	\section{Definitions}
	\notes{
		\begin{itemize}
			\item \hl{\textbf{Algebra}} := ``Reunion from broken parts"\\
			\item \hl{\textbf{Abstract Algbera}} := ``The study of algebraic objects/groups"
		\end{itemize}
	}
	\section{Groups}
	\notes{
		A \hl{\textbf{group}} is an ordered pair $(G,*)$ where $G$ is a set and $*:G\times G\to G$ such that the properties are held: 
		\begin{itemize}
			\item \hl{\textbf{Associativity: }}$(a*b)*c=a*(b*c)$
			\item \hl{\textbf{Identity: }}$\forall e\in G$ such that $e*a=a*e=a\forall a\in G$ (e is the identity element)
			\item \hl{\textbf{Inverses: }}$\forall a\in G, a^{-1}\in G$ such that $a*a^{-1}=a^{-1}*a=e$. 
		\end{itemize}
	}
	\remarks{
		If a group is commutative, (that is, $a*b=b*a$), then the group is called \hl{\textbf{abelian}}
	}\\
	\vskip 0.2in
	\examples{
		\textbf{Example 1: }Let $\mathbb{Z}=$ set of integers closed under addition (+). Note that: 
		\begin{itemize}
			\item Note that this is closed under associativity. Let $a,b,c\in \mathbb{Z}$: 
			\begin{gather}
				(a+b)+c\stackrel{?}{=}a+(b+c)\\
				a+b+c\stackrel{?}{=}a+b+c\\
				0=0
			\end{gather}
			This is true, this group is associative
			\item Note that this also has an identity $0$ and that $\forall a\in \mathbb{Z}, a+0=a$.\\ 
			\item  Note that $\forall a\in \mathbb{Z}, -a\in \mathbb{Z}\implies a+(-a)=(-a)+a=0$
		\end{itemize}
		This is a group because this is associative, has an identity, and every element has an inverse. This is also \hl{abelian} because $a+b=b+a, \forall a,b\in \mathbb{Z}$. \\
		\vskip 0.1in
		\textbf{Example 2: } In addition, these groups are also abelian: 
		\begin{itemize}
			\item $\mathbb{Q}=$rational numbers under $+$\\
			\item $\mathbb{R}=$real numbers under $+$\\
			\item $\mathbb{C}=$complex numbers under +
		\end{itemize}
		\vskip 0.1in
		\textbf{Example 3: } However, $\mathbb{Z}$ under $\times$ is not a group. This is because this is not closed under inverses (there is no inverse for 0). Note that $\mathbb{Q}$, $\mathbb{R}$, $\mathbb{C}$ are also not integers because of the 0 element. 
	}
	\vskip 0.2in
	\notes{
		Let $Z^{\times}$ be the set of all integers with a multiplicative inverse. For example, a set $\{1,-1\}$ is a finite, abelian group under multiplication. \\
		If $S$ is a set with a binary operation *, which is closed under * and has the identity $e$, let \hl{$S^*$} be defined by $S^*=\{a\in S, s.t.\; \exists a^{-1}\in S s.t.\; a*a^{-1}=a^{-1}*a=e\}$. In other words, we know the following: 
		\begin{itemize}
			\item $\mathbb{Q}^\times = \mathbb{Q}\setminus \{0\}$\\
			\item $\mathbb{R}^\times = \mathbb{R}\setminus \{0\}$\\
			\item $\mathbb{Z}^\times = \mathbb{Z}\setminus \{0\}$\\
		\end{itemize}
		Note that $S^*$ becomes a group because of the existence of inverses. 
	}\\
	\vskip 0.2in
	\section{Euclid's division algorithm}
	\notes{
		Let $a,b\in\mathbb{Z}$, $b>0$. I want to divide $a$ by $b$. 
		\vskip 0.1in
		Then Euclid's division lemma states the following: 
		$$\exists!\; q, r\in \mathbb{Z}, s.t \; a=qb+r, 0\leq r<b$$
		The $\exists !$ means that they exist uniquely. 
	}
	\vskip 0.05in
	\remarks{
		What if it's not unique? Then there exists these two equations: $$\begin{cases}a=q_1b+r_1\\ a=q_2b+r_2\end{cases}$$
		Solving the system, we get the following: 
		\begin{gather}
			\setcounter{equation}{0}
			q_1b+r_1=q_2b+r_2\\
			(q_1-q_2)b=r_2-r_1\\
			q_1-q_2=\frac{r_2-r-1}{b}
		\end{gather}
		Note that the left hand side is an integer. This means that the two remainders that are between 0 and b are both divisible by b. This cannot happen, because $r_2-r_1<b$. This is a contradiction, so this is \hl{always unique}
	}
\end{document}