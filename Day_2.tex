\documentclass[12pt]{report}
\usepackage[dvipsnames]{xcolor}
\usepackage{amssymb,amsmath,graphicx}
\usepackage{hyperref}
\usepackage{tcolorbox}
\usepackage{tikz, appendix}
\usetikzlibrary{shapes,arrows}
\usetikzlibrary{calc}
\usetikzlibrary{positioning, decorations.pathmorphing, shadows,decorations.markings}
\usepackage{pgf,pgfarrows,pgfnodes,pgfautomata,pgfheaps,pgfshade,tikz}
\usepackage{soul}
\parindent=0pt
\parskip=4pt
\textwidth = 7in
\oddsidemargin = -.3in
\topmargin = -.8in
\textheight = 9.4in
\usepackage{empheq}
\usepackage{fontenc}
\usepackage{tikz}
\usepackage{pgfplots}
\usepackage{mathtools}
\usepackage{xfrac}
\usepackage{geometry}
\usepackage{xcolor}
\usepackage{physics}
\parindent=0pt
\parskip=16pt
\usepgfplotslibrary{fillbetween}
\newcommand*\widefbox[1]{\fbox{\hspace{2em}#1\hspace{2em}}}
\newcommand{\remarks}[1]{{\leavevmode\color{blue} #1}}
\newcommand{\examples}[1]{{\leavevmode\color{red} #1}}
\newcommand{\definitions}[1]{{\leavevmode\color{OliveGreen} #1}}
\newcommand{\recap}[1]{{\leavevmode\color{teal} #1}}
\newcommand{\proof}[1]{{\leavevmode\color{purple} #1}}
\newcommand{\Mod}[1]{\(\mathrm{mod}\#1)}
\DeclareMathOperator{\rref}{rref}
\DeclareMathOperator{\nullity}{nullity}
\DeclareMathOperator{\Span}{Span}	
\DeclareMathOperator{\Nul}{Nul}
\parindent=0pt
\parskip=4pt
\textwidth = 7in
\oddsidemargin = -.3in
\topmargin = -.8in
\textheight = 9.4in

\begin{document}
	\begin{center} \textbf{\textsc{Math 171 \ \hfill Day 2 - Congruence and examples of Abelian groups \hfill Abstract Algebra}} \\ \vspace{-.1in} \rule{\linewidth}{.01in} \end{center} \vspace{-.15in} \textsc{Prof. Fukshansky}  \hfill \vskip .2in
	\section{Recap}
	\recap{
		\hl{Euclid's division algorithm:} Let $a,b\in \mathbb{Z}, b>0$. Then $\exists ! q,r\in \mathbb{Z}, s.t.\; a=qb+r, 0\leq r<b$. 
		\vskip 0.05in
		\hl{The algorithm to find the greatest common divisor:}Let $a,b\in \mathbb{Z}$. The $GCD(a,b)$ is a positive integer with the following property: $$\text{If }c|a, c|b\implies c|d, |a|\geq b>0$$ Note that $c|a$ means ``$c$ evenly divides $a$". 
		\vskip 0.05in
		The algorithm is a procedure that allows us to find the greatest common factor between two numbers. These are the steps: 
		\begin{enumerate}
			\item $a=q_0b+r_0, 0\leq r_0<b$. If $r_0=0$, then $b$. 
			\item If $r_0>0, b=q_1r_0+r_1, 0\leq r_1<r_0$. If $r_1=0$, then $r_0$. 
			\item If $r_1>0$, $r_0=q_2r_1+r_2, 0\leq r_2<r_1$. If $r_2=0$, then $r_1$. 
			\item Steps 1 to 3 are recursive. Repeat until you get an $r=0$. Note that $r_{n-1}=q_{n}r_n$. 
		\end{enumerate}
	}
	\examples{
		\textbf{Example 1: }a=18, b=7\\
		We can create a table with the steps: \\
		\begin{center}
			\begin{tabular}{|c|c|c|c|}
				\hline
				Step \# & $a=qb+r$ & $q$ & $r$\\
				\hline
				1 & 18=2*7+4 & 7 & 4\\
				\hline
				2 & 7=4*1+3 & 4 & 3\\
				\hline
				3 & 4=3*1+1 & 3 & 1\\
				\hline
				4 & 3=3*1 & 3 & 0\\
				\hline
			\end{tabular}
		\end{center}
		\begin{empheq}[box=\widefbox]{gather*}
			GCD(18,7)=1
		\end{empheq}
	}
	\remarks{
		\textbf{Consequences: }Given an $a,b\in\mathbb{Z}$, there exists integers $x$ and $y$ such that $GCD(a,b)=ax+by$. In our previous example, we found that the GCD is 1. Therefore, 1 can be written as a linear combination of 18 and 7: $$1=18(2)+7(-5)$$
	}
	\section{Groups and examples of abelian groups}
	\definitions{
		\textbf{Definition: }2 integers $a,b$ are said to be \hl{Congruent} to each other modulo $n$-positive integers. $$a\equiv b\pmod n$$ if $n|a-b$.
	}
	\vskip 0.1in
	\examples{
		\textbf{Example: }
		$7\equiv 1 \pmod 3\implies 3|7-1$\\
		Note that this is true, because 3 evenly divides 6. 
	}
	\vskip 0.1in
	\definitions{
		\textbf{Definition: } A relation $\sim$ ion a set $S$ is called an \hl{equivalence relation} if it satisfies the following: 
		\begin{enumerate}
			\item \hl{Reflexive condition:} $\forall a\in S, a\sim a$. 
			\item \hl{Symmetric condition:} If $a\sim b, b\sim a$. 
			\item \hl{Transitive condition:} If $a\sim b, b\sim c$, then $a\sim c$. 
		\end{enumerate}
	}
	\examples{
		\textbf{Example:} Prove that $\equiv\pmod n$ is an equivalence relation. 
	}
	\vskip 0.1in
	\proof{
		We must prove the three conditions: 
		\begin{enumerate}
			\item \textbf{Reflexive: }$a\equiv a\pmod n$ This is true because 0(a-a) divides n evenly. 
			\item \textbf{Symmetric: }$a\equiv b\pmod n$, $b\equiv a\pmod n$. Let's do a little algbera. We are given that $a-b=0\pmod n$. If we multiply both sides by $-1$, then we get $b-a\equiv 0\pmod n$, and therefore $b\equiv a\pmod n$. 
			\item\textbf{Transitive: }$a\equiv b\pmod n, b\equiv c\pmod n$, $a\equiv c\pmod n$. Again, we can do a little algebra to see this is true: 
			\begin{gather}
				n|a-b, n|b-c\\
				n|(a-b)+(b-c)\\
				n|a-c \implies a\equiv c\pmod n
			\end{gather}
		\end{enumerate}
	}
	\definitions{
		\textbf{Definition: } For any element $a\in S$, its \hl{equivalence class} under $\sim$ is: $$\left<a\right>=\{b\in S, a\sim b\}.$$
		If we let $a,c\in S$, this implies either $\left<a\right>=\left<c\right>$ or $\left<a\right>\cap\left<b\right>=\emptyset$. \\
		This means that $S$ can be represented as a disjoint union of equivalence classes under $\sim$. 
	}
	\vskip 0.1in
	\examples{
		\textbf{Example: }Congruence modular 5\\
		Let $a\in \mathbb{Z}\implies a=5q+r, 0\leq r<5$\\
		This means that $a\equiv r\pmod 5$. This congruence class is $\{0,1,2,3,4\}$. Hence each value of the remainder $\{0,1,2,3,4\}$ defines a congruence class $\pmod 5$. Furthermore, for any $0\leq r_1\neq r_2<5$, since $0\leq |r_1-r_2|$, $5\not |(r_1-r_2)$.  
	}
	\vskip 0.1in
	\definitions{
	\textbf{Lemma: } Congruence classes of integers $\pmod n$ are given by the possible values of the remainder under division by $n$. $0\leq r<n-1$. \\
	\textbf{Definition: } Let $a,b\in\mathbb{Z}, st\; 0\leq a,b<n$. \hl{Addition $\pmod n$} is defined as: 
	$$a\bigoplus_n b=(a+b)\pmod n$$\\
	\textbf{Definition: }The set of congruence classes of integers $\pmod n$ with $\bigoplus_n$ is denoted \hl{$\mathbb{Z}/n\mathbb{Z}$}. It is read as ``Z mod n Z". Hence $\mathbb{Z}/n\mathbb{Z}$ can be identified with $\{0,1,...,n-1\}$
	}
	\vskip 0.1in
	\definitions{
	\textbf{Proposition:} $\mathbb{Z}/n\mathbb{Z}$ is an abelian group. 
	}
	\vskip 0.1in
	\proof{
		This operation is associative by $\bigoplus_n$ because it works the same way as it does in $\mathbb{Z}$. This is also commutative because + is commutative in $\mathbb{Z}$. We just have to prove that an identity exists and inverses also exist. 
		\begin{enumerate}
			\item \textbf{Identity: }The identity is 0. This is because $\forall a\in \mathbb{Z}/n\mathbb{Z}$, $a\bigoplus_n 0=0\bigoplus_n a=a$. 
			\item \textbf{Inverses: }Let $a\in \mathbb{Z}/n\mathbb{Z}$. Then $a^{-1}=n-a$ because $n-a+a\pmod n\equiv n\pmod n\equiv 0\pmod n$. 
		\end{enumerate}
	}
	\vskip 0.1in
	\recap{
		We constructed an infinite fmaily of finite abelian groups [modulo]: 
		\begin{itemize}
			\item For each $n\in \mathbb{Z}$, $\mathbb{Z}/n\mathbb{Z}$ is an abelian group under $\bigoplus_n$. \\
			\item Multiplication is the same, but it is a little harder
			\item NOT ALL BINARY OPERATIONS ARE ASSOCIATIVE. An example is * defined as: $a*b=a^b$. 
		\end{itemize}
	}
\end{document}